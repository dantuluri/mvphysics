%%%%%%%%%%%%%%%%%%%%%%%%%%%%%%%%%%%%%%%%%%%%%%%%%%%%%%%%%%%%%%%%%%%%%
% LaTeX Template: Project Titlepage Modified (v 0.1) by rcx
%
% Original Source: http://www.howtotex.com
% Date: February 2014
% 
% This is a title page template which be used for articles & reports.
% 
% This is the modified version of the original Latex template from
% aforementioned website.
% 
%%%%%%%%%%%%%%%%%%%%%%%%%%%%%%%%%%%%%%%%%%%%%%%%%%%%%%%%%%%%%%%%%%%%%%

\documentclass[12pt]{report}
\usepackage[a4paper]{geometry}
\usepackage[myheadings]{fullpage}
\usepackage{fancyhdr}
\usepackage{lastpage}
\usepackage{graphicx, wrapfig, subcaption, setspace, booktabs}
\usepackage[T1]{fontenc}
\usepackage[font=small, labelfont=bf]{caption}
\usepackage{fourier}
\usepackage[protrusion=true, expansion=true]{microtype}
\usepackage[english]{babel}
\usepackage{sectsty}
\usepackage{url, lipsum}


\newcommand{\HRule}[1]{\rule{\linewidth}{#1}}
\onehalfspacing
\setcounter{tocdepth}{5}
\setcounter{secnumdepth}{5}

%-------------------------------------------------------------------------------
% HEADER & FOOTER
%-------------------------------------------------------------------------------
\pagestyle{fancy}
\fancyhf{}
\setlength\headheight{15pt}
\lhead{Egg Drop Vehicle Project}
\fancyhead[R]{Surya Dantuluri}
\fancyfoot[R]{Page \thepage\ of \pageref{LastPage}}
%-------------------------------------------------------------------------------
% TITLE PAGE
%-------------------------------------------------------------------------------

\begin{document}

\title{ \normalsize \textsc{}
		\\ [2.0cm]
		\HRule{0.5pt} \\
		\LARGE \textbf{{Egg Drop Vehicle Project}}
		\HRule{2pt} \\ [0.5cm]
		\normalsize \today \vspace*{5\baselineskip}}

\date{}

\author{
		Surya Dantuluri \\
		Monta Vista High School \\
		AP Physics 1}

\maketitle
\tableofcontents

%-------------------------------------------------------------------------------
% Section title formatting
\sectionfont{\scshape}
%-------------------------------------------------------------------------------

%-------------------------------------------------------------------------------
% BODY
%-------------------------------------------------------------------------------

\chapter{Process}
\section{Design Process}
\qquad Initially my idea was to create a big vehicle that would slowly descend because of drag. This big vehicle would maintain stability by rotating and eventually land on the ground. My initial idea, however, exceeded the weight rule of 100 grams. This was because the combination of a heavy and robust base with a large wing span and large wings were too heavy, even when built with lightweight balsa wood. 

This is why for the second iteration, I designed wings that were as large as the ones I previously built for my initial idea yet were significantly lighter. Instead of using cardboard, I used light balsa wood to be the wings. This second idea still was heavy and I was unsure how rigid the wings were to actually rotate the vehicle. This is because the arms of wings could not be properly mounted to the base of the vehicle.

My third and final reiteration of my vehicle incorporated large wings, a large wingspan and a rigid base. These elements gave stability to the vehicle while maintaining weight. More importantly, the third reiteration distributed weight more evenly by thinning the base of the vehicle to allow for a larger wingspan. This is logical since the base is thick enough for ground impact and the conglomerate vehicle isn't stable enough for its size. Moreover, by removing the top section of the base of the vehicle, the base became even thinner, allowing for more space for the arms to set in place, strenthening the wings, giving the vehicle the capability to rotate more. The wings were also reingineered to be made of aluminum foil with minimal balsa wood scaffholding to keep the wings rigid. Lastly, airflow within the base was considered to be an issue with the design, since the base is very large and is opening downward. To combat this, aluminum foil was placed on all sides of the base, forcing air to go through the vehicle, exiting through the top of the vehicle. This reinforces stability since air that exits through the top is faster, thrusting the vehicle downards.

\section{Testing Design}
All three iterations of the vehicle were tested from a height of 2 meters.\newline 


\hspace*{-0.7cm}\textbf{How I tested my ideas}


I dropped them from my neighbor's apartment, which was on the second story. By testing it from a distance of 2 meters, I could understand what could possibly be an issue when dropping it from a far higher height.

Using this method of testing, I found various flaws with my design, causing me to reiterate over the problems and fixing them in the next design.

For example, when testing I found my intial iteration was unstable and heavy. This was a result of the heavy wings. This method of trial on a small scale and error gave me insight on what to fix and what problems may occur on test day.\newline \newline
\textbf{What I learned from each test}
\begin{itemize}
	\item I learned from the first test that the wings were heavy, causing the vehicle to be unstable, ultimately crashing the vehicle.
 	\item From the second test I learned that the arms for the wings weren't supported enough to rotate the vehicle.
\end{itemize}


From these tests, my final iteration worked better than expected and far better than the previous iterations.


\section{Timeline of Design}
I spent an hour thinking and sketching out possible ideas. I used a notebook to sketch 5-10 different ideas and finally decided on one. 

I spent 2 hours collecting materials. In order to get the wood for my project, I had to travel down to a hobby store relatively far away from me.

I spent 3 hours designing the first prototype since I had to sketch each angle for every truss and glue the foundation together. I spent 5 hours desining and prototyping my 2nd iteration of my vehicle design. This was since I needed to build another base which was lighter and had to make significantly ligher wings. I spent 16 hours on my last iteration since I had to individually sand, chip, and drill the base to make it ligher. Moreover, I had to overhaul the design of the wings and find ligher arms. Moreover, I had to design and manufacture the airflow system in the vehicle, to increase stability. Some of the 16 hours was spent waiting for the Cyanoacrylate fast-drying glue to dry and ventilating my workspace.


\section{Materials Used}
\begin{itemize}
  \item Balsa Wood
  \item Bass Wood
  \item Cyanoacrylate (Glue)
  \item Duct-tape
  \item Aluminum Foil
\end{itemize}



\chapter{Drawings}
\textbf{Note: All Drawings are drawn to scale}
\section{Front Side of Vechicle}
\section{Sides of Vehicle}
\section{Top of Vehicle}


\chapter{Photograph}
\section{Front Side of Vechicle}
\section{Sides of Vehicle}
\section{Top of Vehicle}



\chapter{Controlling Vehicle Speed and Rotation}



\section{Speed Control: Linear Dynamics}

\qquad According to Newton's 2nd law of motion, $F_{net} = ma$ where $F$ is force and $m$ is mass and $a$ is acceleration. In this situation, as the vehicle is dropped from a height $h$, air resistance, $ F_{air} $ counteracts the gravitational force, $ F_{g} $. This would mean the net force of the vehicle would be $F_{g} - F_{air}$
Applying Newton's 2nd law, we can deduce
\begin{equation} \label{eq:1}
F_{g} - F_{air} = ma
\end{equation}
 in this situation. From equation \ref{eq:1} we can find $F_{air}$ is directly proportional to negative acceleration, $-a$ since we can't control gravitational force (since it is constant throughout the drop) and mass $m$ is a constant too.
 Since $F_{air}$ is directly proportional to negative acceleration, $-a$, I tried to maximize air resistance as much as possible. This would minimize acceleration and result in a slow descend of the vehicle downward, which is ideal to keep the egg intact. I decided to use large wings to maximize this air resistance force and shape the base of my vehicle in a cone shape to increase air contact, increasing the air resistance force.
% like\textsubscript{this}

%Use kinematics equations to explain how you kept the vehicle from striking the ground at high speed.
\section{Speed Control: Linear Kinematics}
The goal here is to reduce the final velocity, $v_f$ of the vehicle when it hits the ground. To understand how I could accomplish this, I used the kinematics equation $v_f^2 = v_i^2 + 2a\Delta x$ where $a$ is acceleration and $x$ is distance in meters. In this instance, $x$ should be Height $h$ from the height of the drop to the ground, making the equation 
\begin{equation} \label{eq:2}
v_f^2 = v_i^2 + 2a\Delta h
\end{equation}.



\section{Rotation Control: Angular Dynamics}
\section{Rotation Control: Energy Conservation}


\chapter{Maintaining Vehicle Stability}
\section{Statics: Forces and Torques}


\chapter{Controlling Impact with the Ground}
\section{Impact: Linear Dynamics}
\section{Impact: Linear Momentum, Impulse, and Collisions}
\section{Impact: Work}


\chapter{Additional Benefits of Rotation}



\end{document}

%-------------------------------------------------------------------------------
% SNIPPETS
%-------------------------------------------------------------------------------

%\begin{figure}[!ht]
%	\centering
%	\includegraphics[width=0.8\textwidth]{file_name}
%	\caption{}
%	\centering
%	\label{label:file_name}
%\end{figure}

%\begin{figure}[!ht]
%	\centering
%	\includegraphics[width=0.8\textwidth]{graph}
%	\caption{Blood pressure ranges and associated level of hypertension (American Heart Association, 2013).}
%	\centering
%	\label{label:graph}
%\end{figure}

%\begin{wrapfigure}{r}{0.30\textwidth}
%	\vspace{-40pt}
%	\begin{center}
%		\includegraphics[width=0.29\textwidth]{file_name}
%	\end{center}
%	\vspace{-20pt}
%	\caption{}
%	\label{label:file_name}
%\end{wrapfigure}

%\begin{wrapfigure}{r}{0.45\textwidth}
%	\begin{center}
%		\includegraphics[width=0.29\textwidth]{manometer}
%	\end{center}
%	\caption{Aneroid sphygmomanometer with stethoscope (Medicalexpo, 2012).}
%	\label{label:manometer}
%\end{wrapfigure}

%\begin{table}[!ht]\footnotesize
%	\centering
%	\begin{tabular}{cccccc}
%	\toprule
%	\multicolumn{2}{c} {Pearson's correlation test} & \multicolumn{4}{c} {Independent t-test} \\
%	\midrule	
%	\multicolumn{2}{c} {Gender} & \multicolumn{2}{c} {Activity level} & \multicolumn{2}{c} {Gender} \\
%	\midrule
%	Males & Females & 1st level & 6th level & Males & Females \\
%	\midrule
%	\multicolumn{2}{c} {BMI vs. SP} & \multicolumn{2}{c} {Systolic pressure} & \multicolumn{2}{c} {Systolic Pressure} \\
%	\multicolumn{2}{c} {BMI vs. DP} & \multicolumn{2}{c} {Diastolic pressure} & \multicolumn{2}{c} {Diastolic pressure} \\
%	\multicolumn{2}{c} {BMI vs. MAP} & \multicolumn{2}{c} {MAP} & \multicolumn{2}{c} {MAP} \\
%	\multicolumn{2}{c} {W:H ratio vs. SP} & \multicolumn{2}{c} {BMI} & \multicolumn{2}{c} {BMI} \\
%	\multicolumn{2}{c} {W:H ratio vs. DP} & \multicolumn{2}{c} {W:H ratio} & \multicolumn{2}{c} {W:H ratio} \\
%	\multicolumn{2}{c} {W:H ratio vs. MAP} & \multicolumn{2}{c} {\% Body fat} & \multicolumn{2}{c} {\% Body fat} \\
%	\multicolumn{2}{c} {} & \multicolumn{2}{c} {Height} & \multicolumn{2}{c} {Height} \\
%	\multicolumn{2}{c} {} & \multicolumn{2}{c} {Weight} & \multicolumn{2}{c} {Weight} \\
%	\multicolumn{2}{c} {} & \multicolumn{2}{c} {Heart rate} & \multicolumn{2}{c} {Heart rate} \\
%	\bottomrule
%	\end{tabular}
%	\caption{Parameters that were analysed and related statistical test performed for current study. BMI - body mass index; SP - systolic pressure; DP - diastolic pressure; MAP - mean arterial pressure; W:H ratio - waist to hip ratio.}
%	\label{label:tests}
%\end{table}